\documentclass{article}
\usepackage{amsmath}

\title{Banco de Dados e SQL}
\author{Seu Nome}
\date{\today}

\begin{document}

\maketitle

\section{Conceitos e Evolução de Banco de Dados}
Bancos de dados são sistemas responsáveis pelo armazenamento e gerenciamento de dados. Sua evolução inclui:
\begin{itemize}
    \item Arquivos simples
    \item Sistemas hierárquicos
    \item Sistemas de rede
    \item Modelo relacional
    \item Bancos NoSQL
\end{itemize}

\section{SGBD - Sistema Gerenciador de Banco de Dados}
Um SGBD gerencia o banco de dados e garante a integridade dos dados. Exemplos:
\begin{itemize}
    \item MySQL
    \item PostgreSQL
    \item SQL Server
    \item Oracle
\end{itemize}

\section{Arquiteturas de Banco de Dados}
\begin{itemize}
    \item Monolítica: Tudo em um único servidor.
    \item Cliente/Servidor: Banco separado da aplicação.
    \item Distribuída: Dados espalhados em múltiplos servidores.
\end{itemize}

\section{Modelagem Conceitual}
\subsection{Modelo Entidade-Relacionamento (MER)}
Utilizado para representar a estrutura do banco de dados graficamente.

\subsection{Modelo Relacional}
Baseado em tabelas, utilizando chaves para relacionar os dados.

\section{Normalização}
Processo que reduz redundâncias e melhora a estrutura do banco.

\section{Álgebra Relacional}
Conjunto de operações formais para manipulação de dados.

\section{Linguagem SQL}
\subsection{DDL - Data Definition Language}
Usado para definir estruturas:
\begin{verbatim}
CREATE TABLE exemplo (
    id INT PRIMARY KEY,
    nome VARCHAR(100)
);
\end{verbatim}

\subsection{DML - Data Manipulation Language}
Manipulação dos dados:
\begin{verbatim}
INSERT INTO exemplo (id, nome) VALUES (1, 'João');
SELECT * FROM exemplo;
\end{verbatim}

\subsection{DCL - Data Control Language}
Controle de permissões:
\begin{verbatim}
GRANT ALL PRIVILEGES ON exemplo TO usuario;
\end{verbatim}

\end{document}
